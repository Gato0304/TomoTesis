Se comenz� por emular un sistema de segundo orden  que permitiera obtener el comportamiento de la planta, para esto se tomo el sistema de un motor de segundo orden que cumple con las ecuaciones mostradas en \ref{ec:motorarmadura}.

Las cuales se pueden organizar en esta manera:
 \begin{center}
 \begin{equation}
 	\left\{
	\begin{matrix}&\dot{W}=I_{M}\left ( \frac{K_{M}}{J_{M}} \right )-W_{M}\left (\frac{F_{M} }{J_{M}}\right )\\& \dot{I_{M}} = V_{M}\left ( \frac{1}{L_{M}} \right )-I_{M}\left ( \frac{R_{M}}{L_{M}} \right )-W_{M}\left (  \right \frac{1}{L_{M}} )
    \end{matrix}
\right.
		\label{ec:motorarmadura2}%
		%\textup{Ecuaciones de un motor DC controlado por armadura}	
	\end{equation}
\end{center}
 Con el siguiente sistema de ecuaci�n se puede obtener una planta de segundo orden que cumpla con los siguientes par�metros
 \begin{center}
	 \begin{equation}
		 K_{M}=0.0502\frac{Nm}{A}}\\
		 J_{equiv}= 21,4567*10^{-6}\\
		 R_{m}=10.6\Omega\\
		 L_{m}=0.82m H
		 \label{ec:valoresmotorarmadura}%
	\end{equation}
\end{center}

El sistema de segundo orden que se intenta emular es el Quanser la cual cumple con las caracter�sticas anteriores, para as� poder realizar las pruebas necesarias para evaluar el comportamiento de la planta con un controlador real. Al realizar el espacio de estado de la planta se obtiene

\begin{center}
	\begin{equation}
		\left\{\begin{matrix}
			\begin{bmatrix}
				& \dot{W_{M}}\\ 
				& \dot{I_{M}}
			\end{bmatrix}
			=
			\begin{bmatrix}
				& \frac{-F_{M}}{J_{M}} & \frac{K_{M}}{J_{M}} & \\ 
				&  \frac{-K_{M}}{L_{M}}& \frac{-R_{M}}{L_{M}} & \\ 
			\end{bmatrix}
			\begin{bmatrix}
				& W_{M} \\ 
				& I_{M}
			\end{bmatrix}
			+
			\begin{bmatrix}
				& 0 \\ 
				& \frac{1}{L_{M}}
			\end{bmatrix}
			V_{M}\\ 
			Y=\begin{bmatrix}
				1 & 0
			\end{bmatrix}
			\begin{bmatrix}
				W_{M}\\ 
				I_{M}
			\end{bmatrix}
		\end{matrix}\right.	
    \end{equation}
\end{center}

\begin{center}
	\begin{equation}
		\left\{\begin{matrix}
			\begin{bmatrix}
				& \dot{W_{M}}\\ 
				& \dot{I_{M}}
			\end{bmatrix}
			=
			\begin{bmatrix}
				& 0 & 2340 & \\ 
				&  -61,22& -1,293.10^{4} & \\ 
			\end{bmatrix}
			\begin{bmatrix}
				& W_{M} \\ 
				& I_{M}
			\end{bmatrix}
			+
			\begin{bmatrix}
				& 0 \\ 
				& 1220
			\end{bmatrix}
			V_{M}\\ 
			Y=\begin{bmatrix}
				1 & 0
			\end{bmatrix}
			\begin{bmatrix}
				W_{M}\\ 
				I_{M}
			\end{bmatrix}
		\end{matrix}\right.	
	\end{equation}
\end{center}

Para una planta de segundo orden gen�rica se tiene la ecuaci�n \ref{ec:plantasegundo}
\begin{center}
	\begin{equation}
		\frac{Y(s)}{U(s)}= \frac{b_{2}s^{2}+b_{1}S+b_{0}}{a_{2}S^{2}+a_{1}S+a_{0}} 
		\label{ec:plantasegundo}%
	\end{equation}
\end{center}

Como se pudo observar los motores de segundo orden que se van a estudiar tienen los valores de $b_{1}=0$ y $b_{2}=0 $ entonces la ecuaci�n quedar�a 

\begin{center}
	\begin{equation}
		\frac{Y(s)}{U(s)}= \frac{b_{0}}{a_{2}S^{2}+a_{1}S+a_{0}} 
		\label{ec:plantasegundo}%
	\end{equation}
\end{center}

resolviendo las ecuaciones diferenciales se tiene que 

\begin{center}
	\begin{equation}
		a_{2}\ddot{y}+a_{1}\dot{y}+a_{0}y=b_{0}U 
		\label{ec:ecdiferencial}%
	\end{equation}
\end{center}

\begin{center}
	\begin{equation}
	 	x_{1}=y
		\label{ec:solecdiferencial}%
	\end{equation}
\end{center}

\begin{center}
	\begin{equation}
		dot{x_{1}}=dot{y}=dot{x_{2}}
		\label{ec:solecdiferencial1}%
	\end{equation}
\end{center}
\begin{center}
	\begin{equation}
		\ddot{x_{1}}=\ddot{y}=x_{3}
		\label{ec:solecdiferencial2}%
	\end{equation}
\end{center}

Obteniendo el siguiente espacio de estado

\begin{center}
	\begin{equation}
		\begin{bmatrix}
			\dot{x_{1}}\\
			\dot{x_{2}}
		\end{bmatrix}
		=
		\begin{bmatrix}
			0 & 1\\ 
			-\frac{a_{0}}{a_{2}}& -\frac{a_{1}}{a_{2}}
		\end{bmatrix}x+
		\begin{bmatrix}
			0\\ 
			\frac{b_{0}}{a_{2}}
		\end{bmatrix}
		\label{ec:espaciodestado2}%
	\end{equation}
\end{center}