En la actualidad existe la necesidad en la industria de crear t�cnicas que permitan realizar pruebas de validaci�n a los diferentes dispositivos que constituyen una planta de control. Por este motivo, surge un m�todo llamado $"$Hardware in the loop$"$ que consiste en una t�cnica donde a trav�s de un controlador conectado a un sistema de pruebas se simulan casos de la realidad, con el fin de evaluar el desempe�o del controlador. Disminuyendo de esta manera el costo y el tiempo que demanda realizar los distintos escenarios de pruebas en la planta de inter�s; ya que el controlador est� comprendido como un hardware f�sico y los ensayos se realizan sobre una simulaci�n.

Los sistemas que utilizan el m�todo $"$Hardware in the loop$"$ normalmente est�n integrados por un entorno de simulaci�n y una unidad electr�nica de control que representan al dispositivo que se encuentra bajo prueba, de este modo para la simulaci�n se toma la planta siendo un dispositivo que puede ser aut�nomo como una computadora o como un dispositivo esclavo que puede ser un PLC, FPGA o DSP, entre otros, que est� conectado a un sistema superior. La simulaci�n de la planta puede tener opcionalmente un conjunto de actuadores y sensores necesarios para el sistema de control que servir�n para facilitar el an�lisis. Para la unidad electr�nica de control que representa el elemento de control del sistema a lazo cerrado, este recibe las salidas del modelo simulado.

Es importante destacar que para poder obtener resultados que aporten valor es necesario una alta calidad en el software de simulaci�n. Este software debe combinarse con hardware que pueda permitir la inserci�n de fallas y la habilidad de probar escenarios reales.

En este trabajo se utilizar� el m�todo $"$Hardware in the loop$"$ con el objetivo de validar el desempe�o de un sistema de segundo orden, con el inter�s de reforzar los conocimientos de las asignaturas de la unidad docente de control de la Escuela de Ingenier�a El�ctrica de la Universidad Central de Venezuela.