En la actualidad, existe la necesidad en la industria de crear t�cnicas que permitan realizar pruebas de validaci�n a los diferentes dispositivos que constituyen un sistema de control. Por este motivo, surge un m�todo llamado $"$Hardware in the Loop$"$ que consiste en una t�cnica donde a trav�s de un controlador conectado a un sistema de pruebas, con el fin de evaluar el desempe�o del controlador. Disminuyendo de esta manera el costo y el tiempo que demanda realizar los distintos escenarios de pruebas en el sistema de inter�s, ya que el controlador est� comprendido como un hardware f�sico y los ensayos se realizan sobre una emulaci�n.

Los sistemas que utilizan el m�todo $"$Hardware in the Loop$"$ normalmente est�n integrados por un entorno de emulaci�n y una unidad electr�nica de control que representan al dispositivo que se encuentra bajo prueba. De este modo, para realizar la emulaci�n de la planta se puede utilizar una computadora o un sistema embebido como un PLC, FPGA o DSP, entre otros. Este entorno de emulaci�n ser� conectado a otro sistema f�sico que representar� al controlador,
el cual opcionalmente puede tener un conjunto de actuadores y sensores que servir�an para optimizar el sistema de control. 

Es importante destacar que para poder obtener resultados que aporten valor, es necesario una alta calidad en el software de simulaci�n. Este software debe combinarse con hardware que pueda permitir la inserci�n de fallas y la habilidad de probar escenarios reales.

En este trabajo se utiliz� el m�todo $"$Hardware in the Loop$"$ con el objetivo de validar el desempe�o de un sistema de segundo orden, con el inter�s de reforzar los conocimientos de las asignaturas de la Unidad Docente Control de la Escuela de Ingenier�a El�ctrica de la Universidad Central de Venezuela.